\documentclass[10pt]{beamer}
\usetheme[
%%% option passed to the outer theme
%    progressstyle=fixedCircCnt,   % fixedCircCnt, movingCircCnt (moving is deault)
  ]{Feather}
  
% If you want to change the colors of the various elements in the theme, edit and uncomment the following lines
\definecolor{grey}{rgb}{0.52, 0.52, 0.51}

% Change the bar colors:
\setbeamercolor{Feather}{fg=grey!20,bg=grey}
\include{beamerinnerthemeFeather}
% Change the color of the structural elements:
\setbeamercolor{structure}{fg=grey}

% Change the frame title text color:
%\setbeamercolor{frametitle}{fg=blue}

% Change the normal text color background:
\setbeamercolor{normal text}{fg=black,bg=gray!10}

%-------------------------------------------------------
% INCLUDE PACKAGES
%-------------------------------------------------------

\usepackage[utf8]{inputenc}
\usepackage[english]{babel}
\usepackage[T1]{fontenc}
\usepackage{helvet}
\usepackage{lmodern}


%-------------------------------------------------------
% DEFFINING AND REDEFINING COMMANDS
%-------------------------------------------------------

% colored hyperlinks
\newcommand{\chref}[2]{
  \href{#1}{{\usebeamercolor[bg]{Feather}#2}}
}

%-------------------------------------------------------
% INFORMATION IN THE TITLE PAGE
%-------------------------------------------------------

\title[] % [] is optional - is placed on the bottom of the sidebar on every slide
{ % is placed on the title page
      \textbf{Wavelength}
}

\subtitle[$\lambda$-IDE]
{
      \textbf{$\lambda$-IDE}
}

\author[wavelength]
{      Muhammet Guemues \\
	   \textbf{Jean-Pierre von der Heydt} \\
       \textbf{Markus Himmel} \\
       Marc Huisinga \\
       Philip Klemens \\ 
       \textbf{Julia Schmid} \\       
}

\institute[]
{
      
  
  %there must be an empty line above this line - otherwise some unwanted space is added between the university and the country (I do not know why;( )
}

\date{\today}

%-------------------------------------------------------
% THE BODY OF THE PRESENTATION
%-------------------------------------------------------

\begin{document}

%-------------------------------------------------------
% THE TITLEPAGE
%-------------------------------------------------------

{\1% % this is the name of the PDF file for the background
\begin{frame}[plain,noframenumbering] % the plain option removes the header from the title page, noframenumbering removes the numbering of this frame only
  \titlepage % call the title page information from above
\end{frame}}

\begin{frame}{$\lambda$-IDE}{}
	\begin{overprint}
    \onslide<1>\includegraphics[scale = 0.5]{img/startseite}
    \onslide<2>\includegraphics[scale = 0.2]{img/GUI}
    \end{overprint}
\end{frame}


\begin{frame}{Statistiken}{}
	\begin{itemize}
		\item Lines of Code: 10.208
		\begin{itemize}
			\item Java: 9.644
			\item JavaScript: 269
			\item CSS: 295
		\end{itemize}
		\item Anzahl Klassen: 116
		\item Anzahl Packages: 19
		\item Lines of Testcode: 5.809 
		\item Anzahl Git-Commits: 1.032, Tendenz steigend
	\end{itemize}
	
	\begin{overprint}
    \onslide<1>\includegraphics[width = \textwidth]{img/GitCommits}
    \onslide<2>\includegraphics[width = \textwidth]{img/GitCommitsAbgabe}
    \end{overprint}
\end{frame}

\begin{frame}{Teamarbeit}{}
	\begin{itemize}
		\item Organisation und Aufgabenverteilung vom jeweiligen Teamleiter
		\item Abstimmung und Kommunikation
		\begin{itemize}
			\item Slack
			\item Telegramm
			\item Wiki \textit{(Protokolle und Arbeitspläne)}
			\pause
			\item Wöchentliche Treffens
			\item Crunchtimes
		\end{itemize}
		\pause
		\item \textit{Julia kann ansprechen, dass die Jungs sich schon vorher kannten und sie sich gut eingegliedert hat.}
		\item Jadida zur Entwicklung unserer Teamarbeit
	\end{itemize}
\end{frame}


\begin{frame}{Tools und Bibliotheken}{}
	\begin{center}
    \includegraphics[width=0.9\textwidth]{img/alle}
	\end{center}
\end{frame}

\begin{frame}{Review}{Wasserfallmodell}
	\begin{itemize}
		\item Klare Strukturierung und Planung machen sich später bezahlt
			\begin{itemize}
				\item Teammitglieder teilen spezifische Vorstellungen und Ziele
				\item Entwicklungsprozess wird natürlich in einfacher zu handhabende
					Häppchen zerlegt
			\end{itemize}
		\item Strikte Trennung der Phasen ist oft weniger sinnvoll
			\begin{itemize}
				\item Inkonsistenzen im Entwurf werden durch grobe Implementierung
					effizient gefunden
			\end{itemize}
		\item Starker Fokus auf Dokumente fragwürdig
			\begin{itemize}
				\item Haben das Potenzial, als Referenzpunkte zu dienen\ldots
				\item \ldots dies geschieht in unserer Erfahrung jedoch selten
				\item Hoher Maintenance-Aufwand bei Rückkopplung
				\item Pflichtenheft ist das sinnvollste Dokument
			\end{itemize}
	\end{itemize}
\end{frame}

\begin{frame}{Review}{UI-Planung}
UI Design zu Beginn des Projektes
	\begin{itemize}
		\item Features anhand von Use Cases
		\item Rückkopplung zu Design
	\end{itemize}
\end{frame}

\begin{frame}{Review}{UI-Prototyp}
Prototyp während der Pflichtenheft-Phase
	\begin{itemize}
		\item Testimplementierung wichtiger UI Elemente
		\item Einarbeiten in GWT
	\end{itemize}
\end{frame}

\begin{frame}{Review}{Implementieren in Entwurfsphase}
Beginn der Implementierung zentraler Elemente in der Entwurfsphase 
	\begin{itemize}
		\item[$\leadsto$] frühzeitiges Erkennen von 
		\begin{itemize}
			\item Entwurfsfehlern
			\item widersprüchlichen Anforderungen
		\end{itemize}
	\end{itemize}
\end{frame}

\begin{frame}{Review}{Merge-Tag}
Hat erstaunlich gut funktioniert, weil
	\begin{itemize}
		\item Implementierung in Entwurfsphase $\leadsto$ nur kleinere Entwurfsänderungen %klein im Sinne von nicht weittragend, beeinflusst nicht viel Code
		\item Änderungen mit allen Betroffenen frühzeitig abgesprochen $\leadsto$ funktioniert nur in kleinem Team
	\end{itemize}
\end{frame}



{\1
\begin{frame}[plain,noframenumbering]
  \includegraphics[width = \textwidth]{img/eventbus}
\end{frame}}

\end{document}
