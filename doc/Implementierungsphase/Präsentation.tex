\documentclass[10pt]{beamer}
\usetheme[
%%% option passed to the outer theme
%    progressstyle=fixedCircCnt,   % fixedCircCnt, movingCircCnt (moving is default)
  ]{Feather}
  
% If you want to change the colors of the various elements in the theme, edit and uncomment the following lines
\definecolor{grey}{rgb}{0.52, 0.52, 0.51}

% Change the bar colors:
\setbeamercolor{Feather}{fg=grey!20,bg=grey}
\include{beamerinnerthemeFeather}
% Change the color of the structural elements:
\setbeamercolor{structure}{fg=grey}

% Change the frame title text color:
\setbeamercolor{frametitle}{fg=blue}

% Change the normal text color background:
\setbeamercolor{normal text}{fg=black,bg=gray!10}

%-------------------------------------------------------
% INCLUDE PACKAGES
%-------------------------------------------------------

\usepackage[utf8]{inputenc}
\usepackage[ngerman]{babel}
\usepackage[T1]{fontenc}
\usepackage{helvet}
\usepackage{lipsum}
\usepackage{tikz}
\usepackage[export]{adjustbox}
\usepackage{csquotes}

\usetikzlibrary{arrows, arrows.meta, positioning}


%-------------------------------------------------------
% DEFFINING AND REDEFINING COMMANDS
%-------------------------------------------------------

% colored hyperlinks
\newcommand{\chref}[2]{
  \href{#1}{{\usebeamercolor[bg]{Feather}#2}}
}

%-------------------------------------------------------
% INFORMATION IN THE TITLE PAGE
%-------------------------------------------------------

\title[] % [] is optional - is placed on the bottom of the sidebar on every slide
{ % is placed on the title page
      \textbf{Wavelength}
}

\subtitle[$\lambda$-IDE]
{
      \textbf{$\lambda$-IDE}
}

\author[wavelength]
{     
      Muhammet Gümüs \\
      Jean-Pierre von der Heydt \\  
       Markus Himmel \\
       Marc Huisinga \\
       Philip Klemens \\ 
       Julia Schmid   
}

\institute[]
{
      
  
  %there must be an empty line above this line - otherwise some unwanted space is added between the university and the country (I do not know why;( )
}

\date{09. Februar 2018}

%-------------------------------------------------------
% THE BODY OF THE PRESENTATION
%-------------------------------------------------------

\begin{document}

%-------------------------------------------------------
% THE TITLEPAGE
%-------------------------------------------------------

%{\1 % this is the name of the PDF file for the background
% Mein PDF-Viewer bekommt das nicht hin, das Bild im Hintergrund anzuzeigen
{
\setbeamertemplate{background canvas}{\tikz[remember picture,overlay]\node[opacity=0.3] at (current page.center) {\includegraphics[height=\paperheight]{img/Logo.pdf}};}
\begin{frame}[plain]
\maketitle
\end{frame}
}

%\begin{frame}{Content}{}
%\tableofcontents
%\end{frame}

\begin{frame}[plain]
\frametitle{Muss-Kriterien}
\begin{itemize}
\item Eingabe von $\lambda$-Termen
\item Auswertung von $\lambda$-Termen (Normal-Reduktionsordnung)
\item Fehlermeldung bei invalider Eingabe
\item Abbruch der Reduktion
\end{itemize}
\end{frame}

\begin{frame}[plain]
\frametitle{Kann-Kriterien - implementiert (1)}
\begin{itemize}
\item weitere Auswertungsstrategien: Call-by-Name, Call-by-Value und Applicative Order
\item Ausgabe-Formate: Unicode, Syntaxbaum
\item Ausgabeumfang: alle Teilschritte, erster und letzter Schritt und periodische Abstände
\item Export-Formate: Unicode, \LaTeX, Haskell, Lisp
\item Bibliotheken: natürliche Zahlen, Listen, Tupel, Booleans, Y-Kombinator
\item Namensbindung und Kommentare
\end{itemize}
\end{frame}


\begin{frame}[plain]
\frametitle{Kann-Kriterien - implementiert (2)}
\begin{itemize}
\item Schritt-für-Schritt Modus
\item intelligenter Editor (Einrückungstiefen, Klammernmatching, etc.)
\item Übungsaufgaben
\end{itemize}
\end{frame}

\begin{frame}[plain]
\frametitle{Kann-Kriterien - \alert{Permalinks}}
\enquote{Es kann ein Link generiert werden, der beim Aufrufen den gesamten Zustand
der Sitzung wiederherstellt.}
\begin{itemize}
\item ignoriert den Übungsmodus
\end{itemize}
\end{frame}

\begin{frame}[plain]
\frametitle{Kann-Kriterien - \alert{erweiterte Fehlerdiagnostik}}
\enquote{Im Falle syntaktischer Fehler werden dem Nutzer die genaue Position des Fehlers, die
Art des Fehlers sowie Behebungsmöglichkeiten angezeigt und die relevante Stelle im
Editor hervorgehoben.}
\begin{itemize}
\item Position und Art des Fehlers werden angezeigt
\item Behebungsmöglichkeiten und Markierung im Editor fehlen
\end{itemize}
\end{frame}

\begin{frame}[plain]
\frametitle{Kann-Kriterien: \alert{Syntaxbaum}}
\enquote{Es existiert ein Schritt-für-Schritt-Modus, welcher entweder direkt von der Eingabe
oder durch Pausieren einer laufenden Reduktion erreicht werden kann.
Durch Klicken kann ein bestimmter Teilausdruck ausgewertet werden oder die ausgewählte
Auswertungsstrategie einen einzelnen Schritt machen. Diese Auswertungsschritte können rückgängig
gemacht werden.}
\begin{itemize}
\item manuelle Auswertung des Syntaxbaums im Schritt-für-Schritt Modus nicht möglich
\end{itemize}
\end{frame}

\begin{frame}[plain]
\frametitle{Implementierungsplan}
\begin{figure}[H]
%\hspace*{-3cm}
\includegraphics[trim={0, 7cm, 0, 0}, clip, scale=0.4, page=4]{Implementierungsplan/Implementierungsplan.pdf}
\caption[caption]{Der Verlauf der Implementierungsphase: \\\hspace{\textwidth}
Die oberen Balken entsprechen der Realität, die unteren der Planung.}
\end{figure}
\end{frame}

\begin{frame}[plain]

\end{frame}

\begin{frame}[plain]

\end{frame}


\end{document}
