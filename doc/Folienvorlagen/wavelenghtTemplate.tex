\documentclass[10pt]{beamer}
\usetheme[
%%% option passed to the outer theme
%    progressstyle=fixedCircCnt,   % fixedCircCnt, movingCircCnt (moving is deault)
  ]{Feather}
  
% If you want to change the colors of the various elements in the theme, edit and uncomment the following lines
\definecolor{grey}{rgb}{0.52, 0.52, 0.51}

% Change the bar colors:
\setbeamercolor{Feather}{fg=grey!20,bg=grey}
\include{beamerinnerthemeFeather}
% Change the color of the structural elements:
\setbeamercolor{structure}{fg=grey}

% Change the frame title text color:
%\setbeamercolor{frametitle}{fg=blue}

% Change the normal text color background:
\setbeamercolor{normal text}{fg=black,bg=gray!10}

%-------------------------------------------------------
% INCLUDE PACKAGES
%-------------------------------------------------------

\usepackage[utf8]{inputenc}
\usepackage[ngerman]{babel}
\usepackage[T1]{fontenc}
\usepackage{helvet}
\usepackage{lipsum}
\usepackage{tikz}


%-------------------------------------------------------
% DEFFINING AND REDEFINING COMMANDS
%-------------------------------------------------------

% colored hyperlinks
\newcommand{\chref}[2]{
  \href{#1}{{\usebeamercolor[bg]{Feather}#2}}
}

%-------------------------------------------------------
% INFORMATION IN THE TITLE PAGE
%-------------------------------------------------------

\title[] % [] is optional - is placed on the bottom of the sidebar on every slide
{ % is placed on the title page
      \textbf{Wavelength}
}

\subtitle[$\lambda$-IDE]
{
      \textbf{$\lambda$-IDE}
}

\author[wavelength]
{      Jean-Pierre von der Heydt \\  
	   Muhammet Guemues \\
       Markus Himmel \\
       Marc Huisinga \\
       Philip Klemens \\ 
       Julia Schmid   
}

\institute[]
{
      
  
  %there must be an empty line above this line - otherwise some unwanted space is added between the university and the country (I do not know why;( )
}

\date{30.11.2017}

%-------------------------------------------------------
% THE BODY OF THE PRESENTATION
%-------------------------------------------------------

\begin{document}

%-------------------------------------------------------
% THE TITLEPAGE
%-------------------------------------------------------

%{\1 % this is the name of the PDF file for the background
% Mein PDF-Viewer bekommt das nicht hin, das Bild im Hintergrund anzuzeigen
{
%\setbeamertemplate{background canvas}{\tikz[remember picture,overlay]\node[opacity=0.3] at (current page.center) {\includegraphics[height=\paperheight]{img/Logo.pdf}};}
\begin{frame}[plain]
\maketitle
\end{frame}
}
%}


%\begin{frame}{Content}{}
%\tableofcontents
%\end{frame}

\begin{frame}[plain]

\begin{overlayarea}{\textwidth}{\textheight}
\begin{figure}
    \only<1>{\includegraphics[width=\textwidth]{img/Startseite.png}}%lol this comment is important
    \only<2>{\includegraphics[width=\textwidth]{img/Zoom1.png}}%
    \only<3>{\includegraphics[width=\textwidth]{img/Zoom2.png}}%    
    \only<4>{\includegraphics[width=\textwidth]{img/MussKrit_Eingabe2.png}}%
    \only<5>{\includegraphics[width=\textwidth]{img/MussKrit_Ausgabe.png}}%
	\only<1-5>{\caption{Startseite und erste Benutzung der Entwicklungsumgebung}}
	%Voller Ausgabeumfang
    \only<6>{\includegraphics[width=\textwidth]{img/Ausgabe_Full.png}}%
    \only<6>{\caption{Ausgabe der kompletten Auswertung}}
    %Baumdarstellung
    \only<7>{\includegraphics[width=\textwidth]{img/Ausgabe_Tree1.png}}%
    \only<8>{\includegraphics[width=\textwidth]{img/Ausgabe_Tree2.png}}%
    \only<9>{\includegraphics[width=\textwidth]{img/Ausgabe_Tree3.png}}%
    \only<7-9>{\caption{Baumdarstellung der Auswertung}}
    %Schritt-für-Schritt in Baumdarstellung
    \only<10>{\includegraphics[width=\textwidth]{img/Ausgabe_Tree1.png}}%
    \only<11>{\includegraphics[width=\textwidth]{img/Ausgabe_Tree2.png}}%
    \only<12>{\includegraphics[width=\textwidth]{img/Ausgabe_Tree3.png}}%
    \only<10-12>{\caption{Schritt-für-Schritt Auswertung}}
    %Übungsmodus
    \only<13>{\includegraphics[width=\textwidth]{img/exercise_menue_open.png}}%
    \only<14>{\includegraphics[width=\textwidth]{img/Exercise_(1).pdf}}%
    \only<15>{\includegraphics[width=\textwidth]{img/Exercise_(2).pdf}}%
    \only<16>{\includegraphics[width=\textwidth]{img/Exercise_(3).pdf}}%
    \only<13-16>{\caption{Übungsmodus}}
    %Fortgeschrittene Funktionen
    \only<17>{\includegraphics[width=\textwidth]{img/Startseite.png}}%
    \only<17>{\caption{Startseite der Entwicklungsumgebung}}
    \only<18>{\includegraphics[width=\textwidth]{img/exercise_menue_open.png}}%
    \only<18>{\caption{Auswahl einer Bibliothek}}
    %Export von Ausgaben
    \only<19>{\includegraphics[width=\textwidth]{img/Share_Pressed.png}}%
    \only<20>{\includegraphics[width=\textwidth]{img/Share_Export.png}}%
    \only<19-20>{\caption{Teilen von $\lambda$-Auswertungen}}
\end{figure}
\end{overlayarea}

\end{frame}

\end{document}